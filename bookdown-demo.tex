% Options for packages loaded elsewhere
\PassOptionsToPackage{unicode}{hyperref}
\PassOptionsToPackage{hyphens}{url}
%
\documentclass[
]{book}
\usepackage{amsmath,amssymb}
\usepackage{iftex}
\ifPDFTeX
  \usepackage[T1]{fontenc}
  \usepackage[utf8]{inputenc}
  \usepackage{textcomp} % provide euro and other symbols
\else % if luatex or xetex
  \usepackage{unicode-math} % this also loads fontspec
  \defaultfontfeatures{Scale=MatchLowercase}
  \defaultfontfeatures[\rmfamily]{Ligatures=TeX,Scale=1}
\fi
\usepackage{lmodern}
\ifPDFTeX\else
  % xetex/luatex font selection
\fi
% Use upquote if available, for straight quotes in verbatim environments
\IfFileExists{upquote.sty}{\usepackage{upquote}}{}
\IfFileExists{microtype.sty}{% use microtype if available
  \usepackage[]{microtype}
  \UseMicrotypeSet[protrusion]{basicmath} % disable protrusion for tt fonts
}{}
\makeatletter
\@ifundefined{KOMAClassName}{% if non-KOMA class
  \IfFileExists{parskip.sty}{%
    \usepackage{parskip}
  }{% else
    \setlength{\parindent}{0pt}
    \setlength{\parskip}{6pt plus 2pt minus 1pt}}
}{% if KOMA class
  \KOMAoptions{parskip=half}}
\makeatother
\usepackage{xcolor}
\usepackage{longtable,booktabs,array}
\usepackage{calc} % for calculating minipage widths
% Correct order of tables after \paragraph or \subparagraph
\usepackage{etoolbox}
\makeatletter
\patchcmd\longtable{\par}{\if@noskipsec\mbox{}\fi\par}{}{}
\makeatother
% Allow footnotes in longtable head/foot
\IfFileExists{footnotehyper.sty}{\usepackage{footnotehyper}}{\usepackage{footnote}}
\makesavenoteenv{longtable}
\usepackage{graphicx}
\makeatletter
\def\maxwidth{\ifdim\Gin@nat@width>\linewidth\linewidth\else\Gin@nat@width\fi}
\def\maxheight{\ifdim\Gin@nat@height>\textheight\textheight\else\Gin@nat@height\fi}
\makeatother
% Scale images if necessary, so that they will not overflow the page
% margins by default, and it is still possible to overwrite the defaults
% using explicit options in \includegraphics[width, height, ...]{}
\setkeys{Gin}{width=\maxwidth,height=\maxheight,keepaspectratio}
% Set default figure placement to htbp
\makeatletter
\def\fps@figure{htbp}
\makeatother
\setlength{\emergencystretch}{3em} % prevent overfull lines
\providecommand{\tightlist}{%
  \setlength{\itemsep}{0pt}\setlength{\parskip}{0pt}}
\setcounter{secnumdepth}{5}
\usepackage{booktabs}
\usepackage{amsthm}
\makeatletter
\def\thm@space@setup{%
  \thm@preskip=8pt plus 2pt minus 4pt
  \thm@postskip=\thm@preskip
}
\makeatother
\ifLuaTeX
  \usepackage{selnolig}  % disable illegal ligatures
\fi
\usepackage[]{natbib}
\bibliographystyle{apalike}
\IfFileExists{bookmark.sty}{\usepackage{bookmark}}{\usepackage{hyperref}}
\IfFileExists{xurl.sty}{\usepackage{xurl}}{} % add URL line breaks if available
\urlstyle{same}
\hypersetup{
  pdftitle={The Self-Improvement Guide},
  pdfauthor={Author},
  hidelinks,
  pdfcreator={LaTeX via pandoc}}

\title{The Self-Improvement Guide}
\author{Author}
\date{2023-07-21}

\begin{document}
\maketitle

{
\setcounter{tocdepth}{1}
\tableofcontents
}
\hypertarget{introduction}{%
\chapter{Introduction}\label{introduction}}

1 in every 4 Americans is diagonsed with a mental health disorder. With the advent of social media, low confidence and self-esteem is now the norm among young males and females.

Have you ever wished to make a change in your life situation? Have there been times you've wanted to be more productive, wanted to start hitting the gym, wanted to get rid of those negative thoughts and lead a happy life. Have you ever wished for bulletproof confident, to be able to accomplish massive things and take great risks for the fun of it, to be able to have deep meaningful conversations with a friend.

This open-source guide will lead you to all of that, and more. This self-improvement guide is aimed at being your practical handbook for life, to help you improve all facets of your life and be truly fulfilled with it.

By deciding to read this book, you've embarked on a purposeful journey where you will discover your true self and expand your limits. Some of the \textbf{numerous} benefits that this journey offers include:

\begin{itemize}
\tightlist
\item
  Increased mental clarity and motivation to set and accomplish major goals
\item
  The ability to effortlessly push yourself beyond your comfort zone and make progress
\item
  Improved happiness and satisfaction with your life
\item
  Success in multiple aspects of your life
\item
  No more random bouts of negative feelings and hopelessness
\item
  Total control over the course of your life and a feeling of being in the best era of your life
\item
  Being fully content and grateful for everything in your daily life
\item
  The ability to learn, achieve goals, make changes in your life, and do the things you've always wanted to do much more easily and quickly than the average person
\end{itemize}

But don't just take my word for it. Me along with many others have found true success and happiness through this journey, and my goal with this website is to guide you along the same path I took in order for you to attain the same treasures everyone's seeking.

Below are a few testimonials from other fellow travelers who've walked the same path and attained similar success.

\begin{itemize}
\item
  ``Life before self-improvement for me wasn't great. I'd spend gloomy days in the darkness of my room, dreading to do anything since I was so devoid of motivation. This made me sluggish and hopeless. I wanted to get things done, feel a sense of accomplishment and just be happy, but all I could do was sit motionless on my couch all day and watch YouTube non-stop while eating junk food. I had dreams of building a great physique, learning martial arts, and transforming myself\ldots{} and although I told myself that I wanted to do these things, I continued wallowing in the miserable comfort of my home without taking any steps towards my goals. I felt as if I was tied down and not able to take any steps forward.
  But when I got on self-improvement, things slowly began changing. I finally had a direction in my life, a road map ahead of me to follow. I didn't work towards anything previously since deep down I had confirmed it in my mind that happiness was a myth and nothing could be done to help me. But simply the act of doing things like meditating and working out showed me that there was so much more to life. One can only understand how liberating and vibrant life can get once they get on this journey. I was able to effortlessly quit my life of cheap gratification and instead pursue one that kept me happy and hopeful for the future 24/7. And I know there are still millions of people out there who faced the same conditions as me, and although they still might not be moved by my story, let me tell you that change can be made, all you need is someone to guide you along.'' - Anonymous
\item
  ``I used to be sad and depressed all the time, but since 2020, I've been working on myself; I've devoted myself to a mission, and so I've seen enormous gains in my life. Over this time, I've been able to: sit down doing nothing for upto 3 hours, kill social media FOMO, destroy addictions / neuroticism i used to have, become more confident in my abilities, read plenty of great books, not be so weak and puny lol, meet and network with some amazing people, learn French, build some pretty cool projects/apps.
  And all just by cutting out the stupid bs in my life, and instead focusing on what matters to me. Today, I'm running a business and I'm motivated to work every single day.'' - Aban
\item
  ``I lost 40kg (80+ Pounds), improved my mental health through meditation, journaling and practicing gratitude. That way I got rid of my dark depression and bad anxiety. I read over 40+ books in one year and learned a lot in all different areas of life. I also started different hobbies and was open for new things and experiences. Today I'm the owner of the official Self-Improvement Germany Discord server and my team and I try our best to help out our brothers and sisters which are struggling with life, like I ways, just a few years ago.'' - KazuyaHH
\end{itemize}

\hypertarget{self-improvement}{%
\chapter{Self Improvement}\label{self-improvement}}

Self Improvement may seem like a mystical solution-to-all-your-problems, as it seems to have the ability to provide unimaginable success and totally transform your life\ldots{} which is exactly what it did for me and numerous other indviduals.

But what exactly is Self Improvement? Before setting off on any journey, we must have a plan and general idea of where we're going, along with why we're going there. I'm going to start off by going over what Self-Improvement is and what it isn't, along with some more basic information to clear the basics.

\hypertarget{what-is-self-improvement}{%
\section{What is Self Improvement}\label{what-is-self-improvement}}

Simply put, Self Improvement is a set of habits, practices and mindsets aimed at improving \emph{(well obviously)} every aspect of your life. Anything and everything that can help improve any aspect of your life is a part of this program, and so, embarking upon this journey primes you for success in multiple ways. The core purpose of Self Improvement, in my mind, is to enable one to be completely happy with their life and have the ability to work on improving anything, to work towards developing any skill, or to work on any difficult task, while enjoying and wholeheartedly accepting every bit of it.

\hypertarget{what-self-improvement-isnt}{%
\section{What Self Improvement isn't}\label{what-self-improvement-isnt}}

Although Self Improvement allows you to make unimaginable progress and totally reinvent yourself, this doesn't mean that all of these benefits will come immediately. Like anything, even this process takes time and effort\ldots{} although applying some of the later mentioned principles do make the whole ordeal a lot easier and a lot more fun.

However, don't expect the journey to turn into a God who no longer makes mistakes, or someone who no longer has room to grow or develop whatsover. Upon spending quite some time on this journey, you will soon come to the realization that everything and everyone always has the potential to improve and exceed its limits. Self Improvement in this sense, is more of a lifelong journey than a one time ticket to absolute, complete success. However, rest assured that you will find the lifelong journey quite fulfilling, as it will always keep you occupied with some new challenge to overcome or some new goal to reach, which as you will come to learn is quite intrinsically rewarding.

\hypertarget{what-self-improvement-will-do-for-you}{%
\section{What Self Improvement will do for you}\label{what-self-improvement-will-do-for-you}}

Like I've mentioned a million times already, Self Improvement gives you the power to transform your life and shape it as you desire. Above any and all kinds of improvement you can make and progress you can expect to see, I think the biggest benefit of Self Improvement is that it puts things into perspective and gives you total control over your life. Having total control over your life allows you to work towards improving practically every aspect of your life with ease, and even bypass cravings and negative emotions that keep you stuck where you don't want to be.

By having such control, you gain the ability to turn yourself into EXACTLY whom you want to be: your physique, your personality, the activities that you do, how people view you and even your mindset and the thoughts you think come under your direct control. Think about this for a while. With time and the right kind of effort, you can go from being a shy, nervous little guy who plays video games all day and spends no time outside, to a confident, extroverted, social-butterfly-type person who commands respect among his peers and is treated like a leader by them.

This may sound outlandish to you, or you might think it's possible for others but not for yourself. However, trust me when I say this, if you simply \textbf{BELIEVE} in your ability and in the effectiveness of this guide, and follow this practical guide fully, you will see major progress and positive changes in your life that you never would have anticipated, even in your wildest dreams. So, join me on this journey and allow your upcoming progress to surprise you and those around you.

\hypertarget{a-warning}{%
\section{A Warning}\label{a-warning}}

I need to make this clear before we get into the practical stuff. Self Improvement is notable for a specific \textbf{negative side-effect}, so I'm mentioning it right here in order to make sure you don't fall down this hole.

On this journey you will see yourself transform into a much better self; you will have a lot less worries, a lot of things to be proud of, and overall a great sense of confidence in yourself and your abilities. Do not allow this success to blind you and inflate your ego. Considering yourself a God and above everyone else is the biggest pitfall people fall into while on this journey, and this has two major problems.

\textbf{One}, people like you less for it, and trust me when I say that people can sense how you view them as a person. Someone who views him/herself as better than others is not someone who can easily garner respect and be liked by others, and this cuts you off from being able to establish deep social connections sometimes. By all means tell others about the journey and about the progress you've made, but don't assume a sense of superiority due to it. Not only does it make you less likeable of a person, but it also creates a vulnerable need to feed the narrative that you're better than everyone, eventually leading to feelings of insecurity. \textbf{Two}, it blinds your potential for progress and keeps you stuck where you are. If you regard yourself to already be better than everyone else, you will no longer want to improve yourself, and so this kills any further progress and any fulfillment you would've gotten from that journey.

So recognize that all you're doing on this journey is advancing more of your potential as a human being, and that doesn't inherently make you \emph{better} than someone else. The process of developing and improving ourselves is aimed at learning new things about ourselves and the world around us, and the sense of purpose gained from that is what makes life enjoyable, not the act of becoming better than everyone else.

\hypertarget{the-core-pillars}{%
\chapter{The Core Pillars}\label{the-core-pillars}}

Before stepping out and covering some territory, we must first take a look at the Core Pillars that we will work on improving during our journey. These core pillars are as follows:

\begin{itemize}
\tightlist
\item
  Mental Health
\item
  Physical Health
\item
  Social Skills
\item
  Academics
\item
  Work
\end{itemize}

Now, lets briefly go over each of the five core pillars.

Mental Health is by far the most important pillar, and by far working on improving this pillar will have the greatest return on investment for you. It's poor mental health that makes you unable to be productive, unable to be happy with your life-situation, unable to be motivated, and ultimately unable to make progress and lead the kind of life you wish to live.

By improving our mental health, we make it much easier for ourselves to improve each of the other four pillars.

Next on the list is Physical Health. Your physical health also determines things like your degree of motivation and drive in everyday tasks to some extent, but the biggest benefits that comes with improving it are being more active and healthy, being more comfortable during every-day activities, as well as looking better. Things such as improving your attractiveness and aesthetics also come under physical health, and the specifics regarding these will be covered ahead.

After that we have Social Skills. Improving your social skills is crucial to making more friends and having strong relationships with people in general. Working on these allow you to attain more fulfillment in social settings \emph{(clearly)}, as well as being more likeable and respectable among people. Working on social skills includes things like working on your charisma \emph{(charisma is a skill that can be learned)}, learning to be more likeable, having deeper conversations with others, as well as being more persuasive.

Finally, we have Academics and Work. These pillars are pretty self-explanatory. The key for academics is to use effective techniques that help us minimize study time while still learning effectively in order to get good grades and keep up with the coursework. Work can be a little more complicated, and depending on whether you're working/or planning to work a 9-to-5 or going down the entrepreneurship route, things can change a little, but all of this will be covered in more detail later.

\hypertarget{the-mindsets}{%
\chapter{The Mindsets}\label{the-mindsets}}

Finally, the last set of things to go over before you get into the practical stuff is going to be the mindsets to follow. Having the right frame of mind is rather important to be motivated to make progress as well as to enjoy the journey ahead. We will take a look at some mindsets to adopt and things to keep in mind before we begin.

\hypertarget{the-growth-mindset}{%
\section{The Growth Mindset}\label{the-growth-mindset}}

Understanding and adopting this mindset is crucial to your success on this journey. It is basically the idea that every goal that you want to achieve can be broken up into skills that can be acquired: if you want to become a social-leader-type person who commands respect, you can learn to improve your charisma, study leadership skills and the art of persuasion.

Seeing life through this lens effectively turns it into a video game, where you're gaining XP in new skills all the time to aid the goal-achieving process. For instance, your confidence is like a video game skill which can be leveled up by performing certain activities like approaching strangers and talking to them, or complimenting a random person.

Like in a video game, these skills have progressions to them. In the beginning stages of your journey relatively easy tasks will provide you with a lot of XP, but as time goes on and your skills improve, both your abilities and worthwhile challenges will equally scale up. When you're hitting the weight room for the first time, even curling as little as 5-lb dumbbells might result in sizable muscle gains, but after some months you might need to curl 20-lb dumbbells to see an equal improvement in muscle size.

But the good part is that this means you can become good at almost everything given enough practice. Things such as getting top grades, becoming a social butterfly or becoming ripped are challenges that get easier as you gain more XP in their relevant skills. As you gain more and more XP in these skills, the harder challenges start to get a lot easier, helping aid how smoothly your journey goes, and eventually this journey of leveling up your skills becomes immensely fun and rewarding. It's almost like you're playing an actual video game.

Now, don't ever think you can't achieve any of your future goals. Seen from this perspective it becomes clear that achieving your goals is ultimately inevitable. If you're ever playing a video game and get stumped by a challenging level, you don't say to yourself, \emph{``This level is so hard, I'll never be able to complete it.''} You eventually persevere and figure it out, or maybe you watch an online tutorial and finish it. Either way, your goals are attainable in the same way if you simply believe in yourself and continue leveling up.

\hypertarget{progressive-overload}{%
\section{Progressive Overload}\label{progressive-overload}}

We went over how harder challenges get easy with time as you gain more XP and level up. You need to make sure that you constantly keep increasing the bar and going after more difficult challenges though, since the same easy challenges won't give you a similar jump in XP levels like they did back when they were harder.

There's a sweet spot between overly hard challenges and challenges that are too easy. This goldilocks zone is hit when the challenge is high, but so are your skills. The emotion that is felt when you tackle such a challenge is called the ``Flow State'' (also known as ``Being in the Zone''), and is extremely sought after by creative workers, athletes, and others belonging to many other spheres. This is because the Flow State is extremely rewarding: it results in a sense of fluidity between your mind and body, time begins to slow down, you feel at union with the task and your senses are heightened.

Try your best to chase that sweet spot, and continue to level up your skills as well as raise the challenge that you decide to face whenever you're doing a task or working towards a goal.

\hypertarget{taking-initiative}{%
\section{Taking Initiative}\label{taking-initiative}}

The primary way to level up your confidence skill is to take initiative and move outside your comfort zone. Do you feel anxious in social settings? Go up to a random stranger and talk about literally anything. --? --. --? --. No one cares if it ends up being awkward or if you mess up. In fact, the more awkward it feels, the better it is for you since you gain more confidence XP the more uncomfortable it is. This is because, once you break the resistance and do the task, it becomes 2x easier to do the next time, and 2x more the time after that, and so on.

Action cures fear. By actively choosing to move out of your comfort zone you get rid of the fear that comes with thinking about that activity. This opens up a lot more doors and allows you to do things that you could only previously dream of. And yes, it will be hard in the beginning, and you'll be tempted to not do it, out of fear, out of anxiety, out of compulsion. But everytime you break out of that mysterious force and take action, it's like you gain 5 whole levels in this skill. Taking initiative allows you to regain control over your life and prevents fear from dictating your life choices. And the best part is that right after you do such an uncomfortable activity, massive waves of relief and satisfaction wash over you like you've never felt before.

Given enough time on this journey, your commitment will force you to take the leap and move out of your comfort zone often. It is only by doing so that you can progress in certain aspects, since it destroys our biggest opponent on this journey, \textbf{the resistances of our own mind}. Eventually, the feeling of fulfillment provided by moving out of your comfort zone will result in you seeking its thrill at every oppurtunity. It's almost exactly like riding a rollercoaster. At first you're scared of even the mere thought of going on one, but given enough rides, you soon develop a deep enjoyment for them and can't wait for your next ride on one.

\hypertarget{the-power-of-belief}{%
\section{The Power of Belief}\label{the-power-of-belief}}

Something that becomes more apparent the longer you are on this journey, is the fact that negative thought patterns and emotions are really the biggest obstacle we have on this journey. It's the suppresive beliefs saying that \emph{``You can't do it''}, or \emph{``That's not a realistic goal''} that keep us stuck in mediocrity.

We must learn to harness the power of belief, because if you deeply believe that something can be done, the mind finds ways to do so. This isn't just plain talk, but something that has been scientifically proven. There is true power in belief. You've probably heard of the placebo effect \emph{(if not, go look it up)}: the power of belief is strong enough to cure whole diseases and illnesses.

So, truly believe in yourself, and know that your goals can be achieved easily if you wholeheartedly follow this guide and the journey it will take you on.

\hypertarget{the-asm}{%
\section{The ASM}\label{the-asm}}

There is so much power in belief because of a little something known as the Automatic Success Mechanism (The ASM). The ASM is a subconscious mechanism that all humans possess, and it's primary function is goal seeking. By providing it with a well-defined goal, it works hard to make that goal a reality.

By believing that something can be achieved, and deeply invoking the feelings you'd feel if you were to have achieved that goal, the goal is signaled to your ASM and it begins to work towards it. Failures and mistakes at this stage are useful to us, because they tell the ASM what \textbf{NOT} to do, enabling it to correct course and get closer to success next time.

This is the same principle used by homing missiles, and even in basic tasks like learning to pick up a pencil or catch a fly ball.

We will cover the ASM in a lot more detail further down this guide, but for now just keep in mind that you have at your disposal an extremely powerful device that will automatically help you achieve all your goals, and all you need to do is select the goals and believe that you can achieve them. If you want to learn more about this goal-seeking theory and how to implement it in your daily life, you can also read the bestselling book that this is taken from: \href{https://ia601903.us.archive.org/26/items/TheNewPsychoCyberneticsByMaxwellMaltz1/The\%20New\%20Psycho-Cybernetics\%20by\%20Maxwell\%20Maltz\%20\%20\%281\%29.pdf}{Psycho-Cybernetics}, although we will cover the important aspects of it in lesser detail in this guide as well.

\hypertarget{enjoy-it-while-it-lasts}{%
\section{Enjoy it while it lasts}\label{enjoy-it-while-it-lasts}}

Ultimately, beyond all of these mindsets and principles, this is the \textbf{MOST IMPORTANT}. You're now embarking on a life changing journey, and the older you is eventually going to look back at your present efforts with pride and gratefulness.

It's extremely important to savor every moment in the present fully, in order to remember how your it all feels: your goals, your struggles, and you overcoming the struggles in order to continue on stronger. As weird as this sounds, in the future you're going to miss the struggles and small challenges that you might face, as well as the victories that will eventually build up to your success of course. So don't shy away from anything and accept everything that comes your way fully. Know that all your wins and losses will only carry you towards success, and nothing else.

Apart from all that, savoring the present moment fully is the essential ingredient to happiness. Trying to escape from the present and harboring mental resistance is what causes all sorts of negative disorders like depression, anxiety, and so on. By fully accepting everything that comes in your life from here on out, not only will you become resistant to all challenges and lead a fully positive life, but you will also remember with great detail the beginning of your journey that will go on to provide you with unimaginable success.

I know all of this might sound like cheap motivational talk to get you to feel better, but there really is truth in every single word of what I said. If you ever read this guide in the future, you will look back at these words with a sense of nostalgia, finally having a first-hand realization of what they were leading you to.

\hypertarget{the-2-week-challenge}{%
\chapter{The 2-Week Challenge}\label{the-2-week-challenge}}

Okay so it's finally time to get into the practical stuff and actually start working on habits that will level up your skills. We will start off with the 2-Week Challenge. This is a crucial stepping stone to start working on yourself and begin making progress.

This challenge requires you to complete five tasks daily for the next 2 weeks to help level up your core pillar skills. These tasks are:

\begin{itemize}
\tightlist
\item
  Meditation
\item
  Gratitude Journaling
\item
  Exercise
\item
  Reading
\item
  Learning
\end{itemize}

Before going over these tasks, I want you to \textbf{(and this is very important)} take a sheet of paper and number each line from 1 to 14. Now, draw 5 boxes right next to each number, and label the boxes at the top of the page with the 5 tasks mentioned above. Each time you complete a task (and instructions on completing each task will be given below) you get to check the respective box for that day.

Now, all of these tasks might sound really hard and you might not want to do them. However, in reality this challenge is \textbf{A LOT EASIER THAN YOU THINK} if you follow it exactly as I'm going to describe.

All that is required of you in this challenge is to do the \textbf{BARE MINIMUM}. Literally meditating for 5 seconds, or doing one stretch counts as a full meditation and workout respectively, and so you may tick their boxes. Your mind may read this and think, \emph{``There's no way that's actually going to help me. I'm better off not wasting time on this.''} But remember, if you're unhappy with your current life-situation and are reading this book to transform your life, you probably shouldn't trust your mind, since if your mind was right you'd already be happy with your life and wouldn't need to read this guide.

Our goal with this challenge is to get the ball rolling and make you dip your toes into the core self-improvement practices. At this stage, all your skills are bound to be really low level, since chances are that you haven't worked on levelling up many of these skills before. So, even working on these habits for 1 minute each, or less is bound to help you make progress. If you've never consistently meditated before, even 5 seconds of meditation is crucial progress. Moreover, our main goal with this challenge is to build consistency, which is the key ingredient in levelling up your skills.

Once again, let me make it clear that at this point getting over your mind resistances will be your biggest hurdle. Don't listen to any thoughts that tell you that this won't be helpful. Simply invest a few minutes to this challenge everyday for the next 2 weeks and let your own experiences prove your mind wrong.

\hypertarget{meditation}{%
\section{Meditation}\label{meditation}}

Meditation is crucial to improving our mental health and having a clear perspective on our lives. For now, our goal with meditation is to simply relax and focus on our breath without controlling it. Whenever you notice that your mind has wandered, simply return your attention back to the breath. Each time you do this, you get better at the skill. Think of bringing your attention back as one rep of an exercise, like how curling a dumbbell once counts as one rep.

Eventually, we will want to simply sit back and watch our thoughts and emotions during meditation sessions. Chances are you've never really witnessed your own thoughts before, so you might not even know that that's possible. Although this takes a little more practice, and might not be easy for you within the first two weeks, do know that you'll begin to reach great new insights and experience a massive shift in consciousness once you enter this stage of meditation.

For now, start off meditating for as long as you can, and slowly try to work up to 5 to 10 minutes of meditation. Don't forget however that how long you meditate for doesn't matter as long as you simply do it.

\hypertarget{gratitude-journaling}{%
\section{Gratitude Journaling}\label{gratitude-journaling}}

This habit might feel a little awkward and unnatural at first, but it will have a profound impact on your happiness and outlook towards life. Like the name suggests, gratitude journaling involves writing down one or more things that you're grateful for. Preferrably you would do this on a real sheet of paper with a pen or pencil, but for now you can also type this on a computer or phone if it feels too awkward for you.

This might seem relatively pointless as well, but trust me it has a major positive effect on your thoughts, as it makes you more optimistic and be happier with things that happen in your life. Don't neglect this habit under any condition. If your mind convinces you to do so, try to break through the resistance as that will only help you gain even more XP.

\hypertarget{exercise}{%
\section{Exercise}\label{exercise}}

Exercising has many benefits, but for the purposes of this challenge, the primary benefits it offers are that it helps you get things on track and greatly boosts your happiness. According to a study, exercising seems to have longer-lasting effects and be more effective in improving your mood and emotions than anti-depressants!

The type of exercise you do doesn't really matter at this point. Try lifting weights, taking a walk out in nature, dancing in your room with music on, doing stretches, biking, or really anything else that gets your blood pumping and keeps your whole body active. It doesn't matter if you do it for 1 minute or 1 hour, as long as you simply do it.

\hypertarget{reading}{%
\section{Reading}\label{reading}}

For the reading section of this challenge, we will start off by going over a very specific book. This is a classic self-improvement book which will greatly upgrade your outlook towards a lot of things, and over many insights on human nature. The book is: \href{https://www.google.com/url?sa=t\&rct=j\&q=\&esrc=s\&source=web\&cd=\&ved=2ahUKEwjj9uXHgJqAAxXPM0QIHYboCTEQFnoECBEQAQ\&url=https\%3A\%2F\%2Fwww.researchgate.net\%2Fprofile\%2FRzger-Abdula\%2Fpost\%2FWhat_leadership_books_can_you_recommend_for_me\%2Fattachment\%2F5bae7b2fcfe4a76455f6c7c2\%2FAS\%253A675912282550277\%25401538161454808\%2Fdownload\%2FThe\%2BMagic\%2Bof\%2BThinking\%2BBig.pdf\&usg=AOvVaw0qdaTKW3aTys7FGpy8x7F3\&opi=89978449}{The Magic Of Thinking Big}

Once again--and I know I've mentioned this a dozen times already but I can't stress it enough--all you need to do is read just a couple sentences of the book every day, and that's enough for you to tick the box.

\hypertarget{learning}{%
\section{Learning}\label{learning}}

For learning, we're going to put you on a special 10-week (or 6-week) course that's aimed at teaching you practical ways to be happy and boost your well-being.

This course was conducted by Yale professor Dr.~Laurie Santos, and is currently the highest enrolled course on Coursera. That only goes to show how effective the course is. Work towards watching all the lectures in any given week, and implementing its practical steps.

\href{https://www.coursera.org/learn/the-science-of-well-being/home/week/1}{The Science of Well-Being Course}\\
\href{https://www.coursera.org/learn/the-science-of-well-being-for-teens/home/week/1}{The Science of Well-Being for Teens Course}

\hypertarget{thinking-about-goals}{%
\chapter{Thinking about goals}\label{thinking-about-goals}}

Goals are essential to success because they provide us with a direction to follow. If we want to attain total success, but don't have goals in place to know exactly what that success concretely looks like, we will make little progress because we only have a vague idea of what we want.

Moreover, goals have been proven to been essential to keep us functioning and feel a sense of purpose. We have a feeling of total attunement with our environment when we are totally locked in and working towards making our goals a reality. This is known as the ``Flow State'', and it is one of the most blissful and \emph{complete} states of consciousness known to man.

Continually working towards our goals by spending our times purposefully and enjoying every minute of goal chasing not only makes us feel fulfilled and filled with energy during the present, but also keeps us hopefully and \emph{nostalgic} for the future, since we know we will be able to carve a better future for ourselves by simply doing what we currently love.

This is the beauty of this journey, and I wish this state of enlightenement and total consciousness to everyone reading this. Know that this is possible to attain. Realize that millions of people who were defeated and washed away by life got back up only to realize this very dream, simply by holding on and believing in themselves.

\hypertarget{how-to-set-and-achieve-goals}{%
\section{How to set and achieve goals?}\label{how-to-set-and-achieve-goals}}

Now this sounds great and all, but how do we actually go about setting purposeful goals and enjoying the goal chasing process.

We will start off by imagining the ideal future. The best case scenario. Imagine where you wish to be 10 years from now, or heck, maybe even imagine your perfect future 1 month from now. Whatever it is, let your imagination run wild and fully capture the extent to which you wish to make a transformation in your life.

Now, identify the goals that would lead you to such a future. For example, let's say you imagined yourself as someone with a super muscular physique relaxing in a big mansion with a bunch of friends. In that case, your goals would include:

\begin{itemize}
\tightlist
\item
  Building an attractive, muscular physique
\item
  Becoming rich enough to afford a big mansion
\item
  Having great social skills and a lot of friends
\end{itemize}

Now we will break down each of your goals into 3 aspects to focus on:

\begin{itemize}
\tightlist
\item
  The \textbf{mini-goals} (habits) that will help you reach the goal
\item
  The \textbf{environment} where you will be able to work on these mini-goals
\item
  \textbf{Researching} and learning about the core skills necessary for the goal and mini-goals
\end{itemize}

Now let's take a look at each of these aspects for our first hypothetical goal. For the first goal of building a muscular physique, let's first recognize that this goal falls under the Core Pillar of Physical Health. You will first start by signing up at a gym, or planning to work out elsewhere depending on your preferences \textbf{(Environment)}. Next, for your first workout session, simply get any work done \textbf{(Mini-goals)}. Do 10 pushups and call it a day. Maybe if you're more experienced, plan a basic workout routine and follow it. Just do what you can and what doesn't overwhelm you. Now, slowly begin doing some research about workouts \textbf{(Research)}. Watch YouTube videos about workout splits and what types of workouts to do. You will slowly begin to learn some information on workouts. At the same time, apply your new found knowledge and slowly upgrade your workouts.

As you can see, we begin by straight up jumping into the action, even if you might not know what exactly to do. This is okay, and the truth is, \textbf{no one knows what exactly to do in the beginning.} By simply showing up and doing any work, you start to build consistency. This, coupled with watching a YouTube here and there about workouts when you're bored, slowly starts to educate you on the topic, and in no time you'll become an expert on everything about workouts. You will slowly want to start progressively overloading your habit with time, making sure that you don't feel overwhelmed and that it doesn't feel like a chore. If it does, tone things down for a while. You really don't have to do intense workouts for the rest of your life, and the truth is that almost no one does. All that matters is that you can get at least some work in, but do so consistently. Simply following everything that I've outlined in this page will get you ahead of 90\% of people in the gym, and of course this format and these pricinples apply to every other goal you set.

Let's quickly go over the goal of having a lot of friends as well. If you're in school, well you've already got your environment for you and so you might not need to do anything. If you're an adult, you could try networking in the workplace, or maybe sign up at some local clubs or activities that you're interested in. Next, your mini-goals will be things like introducing yourself to new people, having interesting conversations, organizing to meetups with people in the future, on so on. Some of these may feel awkward at first, but it's a good thing if they do, since that simply means that you're getting over your resistances and gaining a lot of XP in this skill. Awkwardness after doing something uncomfortable is sorta like muscle soreness after a workout. Your body is compensating and trying to recover after doing something it's never done before, but all this means is that after the awkwardness passes, you'll be levelled up and able to do such things with ease the next time. Finally, you will need to read the essential social skills books for success in this skill. Here are the best books that'll give you the best return on your investment:
- \href{https://ia801004.us.archive.org/1/items/HowToWinFriendsAndInfluencePeopleBy/How\%20to\%20Win\%20Friends\%20and\%20Influence\%20People\%20by.pdf}{How to Win Friends and Influence People}
- \href{https://docdro.id/y8ylh73}{The Charisma Myth}

\hypertarget{intermediate-habits}{%
\chapter{Intermediate Habits}\label{intermediate-habits}}

I will now go over a few more habits that you might find useful to implement in your daily routine. You may choose to add it to your daily tracker and check boxes every day after completing them, and that's what I'd recommend. Alternatively, you may also make sure to get these tasks done a few times every week.

Either way, make sure you start off implementing them in a way that doesn't make too much of a difference in your daily life: begin doing one of these tasks once per week, then slowly add more days per week at a rate that makes it seem like there really isn't much of a jump in difficulty. After some weeks or months, add another habit to your list in the same fashion, and then rinse and repeat.

\hypertarget{journaling}{%
\section{Journaling}\label{journaling}}

Journaling is like having a personal chat with yourself on paper. You simply jot down your thoughts, feelings, experiences, and goals in a private journal. It's a cool way to understand yourself better and dig deep into your thoughts. Journaling helps you figure out life's ups and downs and sparks your journey of self-discovery and reinvention. It's like a secret hideout where you can explore your inner world and be your true self without any judgment.

Guess what? Journaling goes beyond just writing stuff down. It's honestly life-changing! When you get into the groove of journaling, you become more self-aware, which is like a superpower for personal growth. You start seeing patterns in your thoughts and behavior, helping you set awesome goals for becoming a better you. Plus, it's a great stress-buster! Pouring out your feelings on paper helps you chill out and manage your overwhelming emotions.

\hypertarget{cold-showers}{%
\section{Cold Showers}\label{cold-showers}}

Let's talk about cold showers. These chilly ones that jolt you awake in the morning are more than just an eye-opener. Cold showers have a variety of benefits that can you help you out on your journey.

I know they might seem intimidating and pointless, but they're surprisingly beneficial for self-improvement. Taking that chilly plunge now and then can be a cool way to build mental resilience and embrace challenges head-on. It's like a small win that sets the tone for the day. And hey, your skin and hair will thank you for the refreshing experience too! If you're up for a little adventure and want to level up your game, give cold showers a shot. No pressure, just a chill way to amp up your mornings!

\hypertarget{initiative}{%
\section{Initiative}\label{initiative}}

We talked about taking initiative and how it's super useful in helping you level up your confidence skill. Here we will be covering a planned self-improvement habit to perform everyday which utilizes the principles of taking initiative and turns in into a performable task.

To complete this task, you must take initiative to move out of your comfort zone and take action every single day. It's exactly like what we talked about earlier in the book. Do something that challenges you and results in you destroying a fear.

However, if you're unable to act on any oppurtunity during your day, or simply weren't able to move out of your comfort zone, you still have a chance. You can choose to take initiative by setting an ambitious goal for yourself for the rest of the day, for eg: not touching your phone until tomorrow, and that counts as finishing the task if you follow through with it.

Doing this everyday will honestly result in a massive improvement in your life. You make great amounts of progress each time you do it, so by doing it everyday you're going to increase your progress on this journey by an exponential amount.

\hypertarget{the-process-of-goal-achieving}{%
\chapter{The Process of Goal Achieving}\label{the-process-of-goal-achieving}}

Previously we covered a basic framework to follow for achieving goals. We will now look at a more in-depth explanation that goes over the whole process of goal seeking, and how this knowledge can immensely benefit you as well as allow you to reach goals much easier and quicker.

In this section, we will look at how to use your ASM (Automatic Success Mechanism) to achieve goals much more easily. By the end of this page, you will have the tools and information necessary to achieve almost any goal \textbf{effortlessly} and with zero stress. The word effortlessly isn't just used as an exaggeration. You will literally need to apply almost no conscious effort in order to achieve your goals. Sounds unbelieveable? Well, it's now time to make you believe in it.

\hypertarget{how-the-asm-works}{%
\section{How the ASM works}\label{how-the-asm-works}}

You might know that birds have a survival instinct which lets them migrate safely across thousands of miles during the winter even without any prior knowledge. Similarly, humans also have a survival instinct allowing them to automatically achieve goals which haven't been achieved before. However, the thing that sets us apart from other animals is that we can set our own goals for our goal-striving mechanism to automatically achieve.

All we need to do is supply the ASM with a goal, and it will automatically work towards achieving it. Infants do this all the time as they learn to do things like pick up an object, to walk, and much more. They do so by trying, making mistakes, and eventually correcting course until they succeed.

When one plays a musical instrument, for eg. piano, for the first time, it's very difficult for them to quickly play the notes in a rhythmic way, but with some practice the same difficult task becomes second nature, and they're able to effortlessly play complicated music. The same applies for any other goal seeking situation: learning to shoot hoops, play chess, deliver persuasive speeches in front of an audience, solve the rubik's cube, and so on. By supplying the goal, the ASM gets to work, and with enough practice it's able to slowly chip away at the mistakes and eventually achieve the goal, with no conscious effort on your part.

\hypertarget{goal-communication}{%
\section{Goal Communication}\label{goal-communication}}

To supply a goal to your ASM, you must have a very clear, well-defined image of what it is. You must know exactly what you want to achieve. Writing detailed descriptions of the goal and how you'd feel upon achieving this goal, as well as imagining yourself achieving the goal in great depth are what allow you to communicate this goal to your ASM.

There is immense power in goal visualization. Dr.~Blaslotto at the University of Chicago conducted an experiment to prove this fact. This experiment involved three groups: the first one didn't touch a basketball for 30 days, the second simply visualized hitting free-throws for 30 minutes a day for 30 days, and the third group actually practiced hitting free-throws for 30 minutes a day for 30 days.

At the end of the 30 days, the first group obviously showed no improvement, and the third group showed a 24\% improvement. Surprisingly however, the second group, who had only visualized hitting free throws without any actual physical training, showed a 23\% improvement! Visualizing a successful action with great detail activates the same muscles and parts of your brain that would work if you were actually physically performing that action.

Try to create a detailed picture in your mind of the success you wish to achieve, whether it be being able to confidentally convince a crowd of people, presenting a charismatic speech, or even scoring well on your tests. Then, spend 30 minutes per day for the next 2 weeks replaying a mental movie of yourself achieving this success and performing in your most ideal, perfect way. After 2 weeks, you will inevitably achieve success if you continue to reinforce your goal and don't surrender to negative beliefs.

\hypertarget{the-power-of-relaxing}{%
\section{The Power of Relaxing}\label{the-power-of-relaxing}}

We can eliminate the majority of stress in our daily lives if we simply delegate all of our goals to our ASM and stop caring about their outcomes. If our goals are well-defined and properly communicated to our ASM, success is inevitable and we don't need to care about it. Moreover, caring about the outcome and stressing out about it will actually make things worse: it might cause us to develop negative thoughts and anxieties which will instead tell the ASM that the failure we so fear is actually the goal to be reached, and it might cause us to try and consciously control the goal seeking process.

The process of goal seeking is purely subconscious, and so, trying to exercise conscious control over it will only hurt us. We must learn to delegate goals to our ASM and stop caring or trying to control it, knowing that doing so will only make things worse; we must instead have faith and realize that the ASM has all the power necessary to make our goal a reality.

If negative emotions and anxieties do crop up however, we can simply focus on completely relaxing all of our muscles, since we can't feel any negative emotions if all of our muscles are relaxed. Also, our emotions don't depend on external conditions, but instead our reactions to them. When a telephone rings, we don't have to pick it up; we can always choose to ignore it. Similarly, when we face an unfavorable situation, we can decide to ignore it since it doesn't have any power over us. By choosing not to give into it, we ensure better emotional stability and well-being, allowing us to better handle tasks and work towards goals.

Visualizations can come in handy once again here. Whenever you find yourself stressed or angry at something, you can imagine yourself literally blowing off steam from a steam valve atop your head, or imagine yourself walking into a well-protected safe house that is in a peaceful setting and heavily fortified from all external elements.

These techniques of relaxation are essential to chasing success, as they allow us to better assess our situations, as well as prevent unfavorable situations from being able to have an effect on us.

\hypertarget{watching-the-thinker}{%
\section{Watching the Thinker}\label{watching-the-thinker}}

A crucial realization that is often caused by, but doesn't have to come from mindfulness meditation practice, is the fact that we don't actually think our thoughts. The ``conscious'' thoughts that we seem to think in our head are actually simply another external stimulus that we perceive, just like the sights and sounds around us, as well as the emotions we experience.

By realizing this, we come to learn that we don't have to give much importance to our thoughts or deeply believe in them, because they are simply byproducts of the mind's conditioning, which is aimed at keeping your self-image and its narrative alive. We will also cover this is much greater detail because this is the most important topic in the entire book, and having a deep understanding of this will allow us to essentially be free of negative emotions, destroy all mind resistances at will, and experience greater happiness 24/7.

If we start watching our thoughts from an unbiased perspective whenever they show up, knowing that we don't think them and that whatever they say isn't necessarily the truth, we can make negative, belittling thoughts as well as negative emotions completely powerless. These thoughts and mind resistances are the biggest obstacle to goal seeking, as they have the ability to convince us that we can't reach our goals, preventing us from signalling the goal to the ASM properly.

Whenever you feel negative emotions like fear, anxiety, worry, tension or hear negative, belittling thoughts in your head saying that you can't do something, or that your goal isn't realistic, simply watch the emotions and thoughts from beyond the mind without judging them. Watch them with a sense of curiosity, with the innate knowledge that you are not the thinker of these thoughts, nor are these thoughts necessarily valid. This will instantly dissolve these negative mind structures, allowing you to continue chasing your goals without having to deal with them.

\hypertarget{using-emotional-steam-to-your-advantage}{%
\section{Using Emotional Steam to your Advantage}\label{using-emotional-steam-to-your-advantage}}

It's common to see people who are really good at an activity normally, but who ``choke'' and mess-up around other people because of being under pressure.

This happens because they aren't able to use crisis situations to their advantage. When one feels nervous and tense during an important activity or performance, it is because they've been supplied with increased strength and wisdom to better handle the situation. The heightened emotion that one feels which seems like tension is actually just neutral ``emotional steam''. This emotional steam can either be anxiety or excitement, depending on how it's perceived. The emotion is the same, but how it's viewed makes all the difference.

If the emotional steam is viewed as tension, thoughts may end up reinforcing failure as the goal instead, resulting in\ldots{} failure. If it's viewed as excitement however, and you continue to believe that you can do it, you'll be able to achieve success.

You could also apply the technique of watching your emotions mentioned above in order to reduce the degree of control the emotion has over you, and to minimize its effects. We will later learn about how it might be better in the long run to adapt that instead, since it will prevent emotions from having too much control over you, allowing you to simply accept them for what they are and move on, instantly dissolving the negative feelings they cause. But if you find that technique difficult to practice for whatever reason, simply knowing that your emotional steam is just excitement which will help you achieve your goal will be useful for the goal seeking situations.

A final note about crisis situations and performing under pressure: when you're learning a skill for the first time and allowing your ASM to work towards reaching the goal associated with it, working under pressure might not be the best option. Working without pressure allows your ASM create more general mind maps under a calm mental state, which can be appropriately altered and modified to suit any similar situation in the future. However, if you only work under pressure, it might not allow you to perform the skill/task during other emotional states. We should find safe ways to practice, maybe with simulated, synthetic pressure, so that you can learn effectively, and make it easier to perform during the actual situation. For example, this is what boxers do by practice punches in front of a mirror (shadow boxing).

\hypertarget{the-winning-feeling}{%
\section{The ``Winning Feeling''}\label{the-winning-feeling}}

When we worry about a situation going wrong, we invoke the same emotions that we would feel if that situation actually went wrong. If you continue to worry and dwell on negative emotions, the situation going wrong will become the goal and your ASM will act in accordance to those emotions. However, if you evoke feelings of success and fully capture it, you will act successfully.

You can reimagine past moments of success you've had in great detail, to fully evoke within you the feeling of success you felt once again, and then with this ``winning feeling of success'' imagine vivid pictures of your new goal, imagining how you would act and feel if you have already succeeded. Delving deeply into this winning feeling and visualizaing your new goal at the same time more effectively communicates the goal to your ASM, allowing you to achieve the goal a lot quicker.

\hypertarget{the-self-image}{%
\chapter{The Self-Image}\label{the-self-image}}

Simply put, your self-image is a mental image of yourself which is formed based on your past experiences and actions. It dictates how you respond to certain events, what is easy for you and what is not, and much more. However, the thing is that your self-image is not fully reflective of what you really can and can't do, and so it simply doesn't show you your \emph{real self} or your \emph{true potential}.

If someone is strong enough to curl 100 lbs, but their self-image says that they're weak, they might only be able to curl upto 80 lbs, no matter how hard they try. It will simply be impossible for them to curl more weight, \textbf{unless they change their self-image.}

\hypertarget{why-is-your-self-image-important}{%
\section{Why is your Self-Image Important}\label{why-is-your-self-image-important}}

Your self-image governs how other people perceive you, how you perceive yourself, and what you can end up accomplishing in life. We generally put way too much faith in our self-images, believing it to be the real us, and so it can be a huge obstacle in our journey of goal seeking. In fact, many of the mind resistances to chasing goals and trying to get things done are a result of an inadequate self-image.

So, it is crucial for us to change our self-image to one that we fully accept and are grateful for, so that we can experience maximum satisfaction and reach those higher states of purposeful work, which requires us to go beyond our limits without being bound by an inadequate self-image.

\hypertarget{consciously-alter-your-thoughts}{%
\section{Consciously Alter Your Thoughts}\label{consciously-alter-your-thoughts}}

You might often have thoughts that tell you what you can and can't do. Your thoughts help reinforce the narrative that you've been believing in all your life, thereby solidfying the apparent validity of your self-image. If your thoughts and self-image have convinced you throughout your life that you're scared of beetles for example, nothing will be able to change that unless some external force causes you to break free of the supposed fear by luck, or if you consciously take the effort to impose a new set of thoughts that affirm that you're not scared of them.

If you wish to get rid of all negative thoughts regarding a particular belief, and instill a new set of positive, uplifting, resassuring thoughts instead, you can do so by consciously thinking these thoughts. By writing down a list of positive \textbf{thoughts of affirmation} to repeat, and then reading from this list out loud or in your head throughout your day, you can slowly convert your subconscious thoughts regarding this belief as well. Eventually you will develop a new belief, formed out of the thoughts of affirmation that you consciously instilled within your mind.

\hypertarget{learn-from-the-experts}{%
\section{Learn from the Experts}\label{learn-from-the-experts}}

Something you can do during your free time is to learn about the lives of those people who have the success you wish to achieve. Watching interviews, reading biographies and learning about the early lives of experts will make clear to you the forces and influences that shaped their choices and personality, allowing them to be where they are today.

By learning from the conditions that primed them for success, you'll be able to fuel your imagination with enough raw material to be able to entertain the idea that change is possible, and that success is inevitable if you effectively believe in it and work towards it.

  \bibliography{book.bib,packages.bib}

\end{document}
